\documentclass[12pt,a4paper]{article}
\usepackage[utf8]{inputenc}
\usepackage[margin=1in]{geometry}


\begin{document}

\begin{center}
{\Large\textbf{Constitution of The Association of Mathematical and Computer Science Students}}

\vspace{0.5cm}
{\large\textbf{2025-2026}}
\end{center}

\vspace{1cm}

\section*{Article I – Name of Organization}

\begin{enumerate}
\item[1.1] The official name of the organization will be – The Association of Mathematical and Computer Science Students.

\item[1.2] The Association of Mathematical and Computer Science Students (AMACSS).
\end{enumerate}

\section*{Article II – Purpose}

\begin{enumerate}
\item[2.1] AMACSS is an independently recognized organization working within the community of the University of Toronto at Scarborough. AMACSS is the officially recognized Departmental Student Association (DSA) for the Department of Computer and Mathematical Sciences at the University of Toronto Scarborough.

\item[2.2] The purpose of AMACSS will be to represent, advocate for and enhance the university experience of students in the Department of Computer and Mathematical Sciences.

\item[2.3] AMACSS will enhance the educational, recreational, social, and cultural environment of the University of Toronto at Scarborough by:

\begin{enumerate}
\item[2.3.1] Fostering a collective vision and purpose among all program students in the Department of Computer and Mathematical Sciences at the University of Toronto at Scarborough

\item[2.3.2] Serving as a comprehensive resource for students in computer and mathematical sciences who need assistance and guidance with their education

\item[2.3.3] Encouraging interaction and the exchange of ideas between students and faculty in the Department of Computer and Mathematical Sciences and working with the Department in order to facilitate long-term growth and improvement.

\item[2.3.4] Acting as a voice for all students in the Department of Computer and Mathematical Sciences at the University of Toronto at Scarborough, addressing and relaying their concerns to the administration, faculty, and other student organizations where appropriate.
\end{enumerate}

\item[2.4] AMACSS fundamentally serves a non-profit function within the University of Toronto at Scarborough, and will not engage in activities that are essentially commercial in nature.
\end{enumerate}

\section*{Article III – Membership}

\begin{enumerate}
\item[3.1] Membership in AMACSS is open to all students, staff, faculty and alumni of the University of Toronto at Scarborough.

\item[3.2] There shall be three (3) categories of membership in AMACSS:

\begin{enumerate}
\item[3.2.1] General Membership

\begin{enumerate}
\item[3.2.1.1] All students enrolled in a Computer Science, Mathematics, or Statistics program at the University of Toronto at Scarborough.

\item[3.2.1.2] All University of Toronto at Scarborough students interested in Computer and Mathematical Sciences.

\item[3.2.1.3] All alumni who were general members during their time as students.
\end{enumerate}

\item[3.2.2] Exclusive Membership

\begin{enumerate}
\item[3.2.2.1] Any person interested in becoming a paid member of AMACSS.

\item[3.2.2.2] Exclusive members pay a mandatory membership fee as determined by the AMACSS team.
\end{enumerate}

\item[3.2.3] Team Membership

\begin{enumerate}
\item[3.2.3.1] Executive Team Members: As defined and regulated in Article IV of this constitution.

\item[3.2.3.2] Non-Executive Team Members: As defined and regulated in Article V of this constitution.
\end{enumerate}
\end{enumerate}

\item[3.3] The term of membership for AMACSS will be from September 1 to August 31 each year.

\item[3.4] Each member shall be afforded the following rights through membership in AMACSS:

\begin{enumerate}
\item[3.4.1] The right to participate and vote in group elections and meetings;

\item[3.4.2] The right to communicate and to discuss and explore all ideas;

\item[3.4.3] The right to organize and engage in activities/events that are reasonable and lawful;

\item[3.4.4] The right to freedom from discrimination on the basis of sex, race, religion, or sexual orientation;

\item[3.4.5] The right to be free from censorship, control, or interference by the University on the basis of the organization's philosophy, beliefs, interests or opinions unless and until these lead to activities which are illegal or which infringe on the rights and freedoms already mentioned above;

\item[3.4.6] The right to distribute on campus, in a responsible way, published material provided that it is not unlawful.
\end{enumerate}

\item[3.5] Each member shall possess the following responsibilities relative to participation in AMACSS:

\begin{enumerate}
\item[3.5.1] Support the purpose of the organization;

\item[3.5.2] Uphold the values of the organization;

\item[3.5.3] Contribute constructively to the programs and activities offered by the organization;

\item[3.5.4] Attend general meetings;

\item[3.5.5] Abide by the Constitution and subsequent official organizational documents;

\item[3.5.6] Respect the rights of peers and fellow members;

\item[3.5.7] Abide by University of Toronto policies, procedures, and guidelines;

\item[3.5.8] Abide by the Laws of the Land, including but not limited to the Criminal Code of Canada.
\end{enumerate}

\item[3.6] AMACSS will collect a mandatory membership fee from each exclusive member each year. This fee will be decided by the AMACSS team after consultations with general members.

\item[3.7] AMACSS values and respects the personal information of its members. AMACSS secures its members' information at all times and will not supply names or other confidential information to third-parties.

\item[3.8] AMACSS will protect the privacy of member information and must use it only for the delivery of service and not for commercial gain.

\item[3.9] Faculty Advisor(s): A number of members of faculty from the Department of Computer and Mathematical Sciences not exceeding 50\% of the number of executives are eligible to serve as Faculty Advisors within AMACSS for a term matching that of the executive that confirmed them under the following terms:

\begin{enumerate}
\item[3.9.1] A member of faculty may be nominated for this role either through self-nomination or by any member of the executive.

\item[3.9.2] Confirmation in the role of faculty advisor shall require a simple majority vote of the executive.

\item[3.9.3] This position is non-exclusive and therefore multiple advisors are possible.

\item[3.9.4] A faculty advisor shall be considered an honorary executive of the association but will not have voting rights.
\end{enumerate}
\end{enumerate}

\section*{Article IV – Executive}

\begin{enumerate}
\item[4.1] The executives of the organization shall include:

\begin{enumerate}
\item[4.1.1] President

\item[4.1.2] Executive Vice-President

\item[4.1.3] Vice-President Academics

\item[4.1.4] Vice-President Finance

\item[4.1.5] Vice-President Outreach

\item[4.1.6] Vice-President Operations

\item[4.1.7] Vice-President Campus Life

\item[4.1.8] Vice-President Special Events

\item[4.1.9] Vice-President Marketing

\item[4.1.10] Vice-President Technology

\item[4.1.11] First-Year Coordinator
\end{enumerate}

\item[4.2] All executive positions are voluntary, non-paid positions open to all undergraduate student members enrolled at the University of Toronto at Scarborough.

\item[4.3] The term for all positions extends from date of hiring of each year until August 31st of the following year.

\item[4.4] The term for First Year Coordinators and Representatives will be from their date of successful hiring until August 31st of the following year.

\item[4.5] The broad responsibilities of each executive position are as follows:

\begin{enumerate}
\item[4.5.1] President

\begin{enumerate}
\item[4.5.1.1] Maintain strong relations with the Department of Computer and Mathematical Sciences faculty, staff and clubs, Department of Student Life, Scarborough Campus Student Union and external entities.

\item[4.5.1.2] Act as the official representative of the association in formal business.

\item[4.5.1.3] Serve as the official signing officer of the association.

\item[4.5.1.4] Enable the transition and continuity of the association from year to year.

\item[4.5.1.5] Fulfill the responsibilities of any vacant executive position or ensure they are fulfilled by another party.

\item[4.5.1.6] Attend all Departmental Student Association Council meetings, except in exceptional circumstances, in which case a designate may be sent.

\item[4.5.1.7] They must not be a current executive of any other Departmental Student Association during their tenure.

\item[4.5.1.8] If they wish to be on a work term during the Fall or Winter semester, a previous agreement with the association Vice-Presidents should be made.

\item[4.5.1.9] He or she is required to implement one (1) new improvement/feature during their term.
\end{enumerate}

\item[4.5.2] Executive Vice-President

\begin{enumerate}
\item[4.5.2.1] Oversee and manage the organization, maintain the integrity of the association, and ensure all events and functions align with the mission of the association.

\item[4.5.2.2] Oversee the internal operations of the organization and manages Vice-Presidents.

\item[4.5.2.3] Oversee the financial and organizational stability of the association.

\item[4.5.2.4] Enable the transition and continuity of the association from year to year.

\item[4.5.2.5] Fulfill the responsibilities of any vacant executive position or ensure they are fulfilled by another party.

\item[4.5.2.6] Attend all Departmental Student Association Council meetings, except in exceptional circumstances, in which case a designate may be sent.

\item[4.5.2.7] They must not be a current executive of any other Departmental Student Association during their tenure.

\item[4.5.2.8] If they wish to be on a work term during the Fall or Winter semester, a previous agreement with the association Vice-Presidents should be made.

\item[4.5.2.9] Serve as President in the event that the office is vacant or the President is otherwise unable to serve.

\item[4.5.2.10] Serve as a secondary signing officer.
\end{enumerate}

\item[4.5.3] Vice-President Academics

\begin{enumerate}
\item[4.5.3.1] Encourage and participate in discourse among students in computer and mathematical sciences regarding their concerns and perspectives with respect to their education and academic programming.

\item[4.5.3.2] Communicate concerns and ideas raised among students in computer and mathematical sciences to the administration, faculty, or other student organizations as appropriate, and work to address those concerns in a proactive and constructive way.

\item[4.5.3.3] Foster an awareness of students' academic rights and entitlements among students in the Department of Computer and Mathematical Sciences.

\item[4.5.3.4] Serve as a point of referral for students in the Department of Computer and Mathematical Sciences, directing them to other resources and services as may be most appropriate to their needs.

\item[4.5.3.5] Act as an official representative of the organization in business pertaining to the academic activities of the Department of Computer and Mathematical Sciences.

\item[4.5.3.6] Responsible for scheduling, planning, preparing, booking, and marketing of review seminars for CMS courses during midterm and final exam seasons

\item[4.5.3.7] Manage and oversee all of the disciplinary representatives.
\end{enumerate}

\item[4.5.4] Vice-President Finance

\begin{enumerate}
\item[4.5.4.1] Oversee the financial resources of the association.

\item[4.5.4.2] Produce monthly financial reports as required by the President.

\item[4.5.4.3] Produce a financial report that includes in-out of financial resources, as well as revenue and expense every four months (or every semester).

\item[4.5.4.4] Provide guidance for the budget of each event.

\item[4.5.4.5] Serve as treasurer and manage the budget.

\item[4.5.4.6] Serve as a signing officer for the bank account.

\item[4.5.4.7] Manage reimbursements.

\item[4.5.4.8] Communicate with the President as necessary.
\end{enumerate}

\item[4.5.5] Vice-President Outreach

\begin{enumerate}
\item[4.5.5.1] Create, update, and maintain sponsorship packages.

\item[4.5.5.2] Find potential sponsors and maintain relationships with external organizations.

\item[4.5.5.3] Organize and run Careers and Advancement events.

\item[4.5.5.4] Serve as a primary point of contact with external organizations if the President or either Vice-Presidents are not available, or if they have already been appointed as the primary point of contact by the President.

\item[4.5.5.5] Responsible for organizing, planning, preparing, and running all the Careers and Advancement events, which may include but not limited to career workshops and networking nights.

\item[4.5.5.6] Attend regular meetings and events as scheduled by the President.

\item[4.5.5.7] Candidate for this position can be registered in one of the Management programs.
\end{enumerate}

\item[4.5.6] Vice-President Operations

\begin{enumerate}
\item[4.5.6.1] Oversee and coordinate the day-to-day operational activities of the organization.

\item[4.5.6.2] Manage room bookings and facility arrangements for all AMACSS events and meetings.

\item[4.5.6.3] Coordinate with other teams to ensure smooth execution of events and activities.

\item[4.5.6.4] Maintain and update organizational procedures and operational guidelines.

\item[4.5.6.5] Serve as a liaison between AMACSS and university administrative departments for operational matters.

\item[4.5.6.6] Manage the Operations Team and oversee logistical support for all organizational activities.

\item[4.5.6.7] Ensure compliance with university policies and procedures for events and activities.

\item[4.5.6.8] Coordinate with the Finance team for budget allocation related to operational expenses.
\end{enumerate}

\item[4.5.7] Vice-President Campus Life

\begin{enumerate}
\item[4.5.7.1] Organize, plan, and run campus life events.

\item[4.5.7.2] Ensure events cater to the social and recreational needs of students.

\item[4.5.7.3] Coordinate with the Marketing team for event promotion and the Outreach team for connections for events.

\item[4.5.7.4] Manage the Campus Life Team.
\end{enumerate}

\item[4.5.8] Vice-President Special Events

\begin{enumerate}
\item[4.5.8.1] Organize, plan, and run special events such as the annual CMS Gala. 

\item[4.5.8.2] Ensure events align with the broader goals of AMACSS of community integrity.

\item[4.5.8.3] Coordinate with the Marketing team for event promotion.

\item[4.5.8.4] Plan major events months in advance by coordinating with the Outreach team and the Co-Presidents.

\item[4.5.8.5] Manage the Special Events Team.
\end{enumerate}

\item[4.5.9] Vice-President Marketing

\begin{enumerate}
\item[4.5.9.1] Promote AMACSS events and seminars through social media and in-class announcements.

\item[4.5.9.2] Assist in the promotion and maintenance of all events pertaining to AMACSS.

\item[4.5.9.3] Maintain communications by responding to messages and maintaining social media accounts.

\item[4.5.9.4] Assist in creating and outreaching sponsorship packages.

\item[4.5.9.5] Manage the Marketing Team.
\end{enumerate}

\item[4.5.10] Vice-President Technology

\begin{enumerate}
\item[4.5.10.1] Oversee the technologies of the association and maintain day-to-day operations.

\item[4.5.10.2] Maintain the AMACSS website, social media pages, and mailing lists.

\item[4.5.10.3] Manage the Technology Team.

\item[4.5.10.4] Responsible for outreach and maintenance of open-source initiatives in the CMS community.
\end{enumerate}

\item[4.5.11] First-Year Coordinators

\begin{enumerate}
\item[4.5.11.1] Serve as a liaison between the executive and first-year students.

\item[4.5.11.2] Inform first-year students about the club's workings.

\item[4.5.11.3] Identify areas of need and communicate them to the executive.

\item[4.5.11.4] Manage the First-Year Team.

\item[4.5.11.5] Assist other directors with various duties as needed.
\end{enumerate}
\end{enumerate}

\item[4.6] From time to time, individual executives may be required to accompany the President to Departmental Student Association Council meetings.

\item[4.7] The executive positions collectively will form a committee that acts as the primary steward of the organization.

\item[4.8] This committee is collectively responsible for the day-to-day decision making of the organization including but not limited to monitoring finances, event planning and execution, member services, and advocating on behalf of members to Administration and student government.

\item[4.9] Any executive of the organization except the President may resign, provided that notice of such resignation is made in writing and delivered to the President. Unless any such resignation is, by its terms, effective on a later date, it shall be effective on delivery to the President, and no ratification by the organization shall be required to make the resignation official.

\item[4.10] The President may resign provided that notice of such resignation is made in writing and delivered to the executive committee at a valid executive meeting. Unless any such resignation is, by its terms, effective on a later date, it shall be effective on delivery to the executive committee, and no ratification by the organization shall be required to make the resignation official.

\item[4.11] Any vacancy of executives shall be filled by the President or designate of the organization until such a time where a by-election is held, a permanent appointment occurs, or a hiring process is conducted.

\item[4.12] Any vacancy of the President shall be filled by another executive committee member as outlined in section 4.5 until such a time as a by-election is held, a permanent appointment, or a hiring process is conducted.

\item[4.13] Executives are required to create year-end reports summarizing their achievements and progress within AMACSS.

\item[4.14] When a meeting is being called, a form must be made available at least 48 hours in advance of the meeting detailing the time, place, topics, and required attendance of the meeting.
\end{enumerate}

\section*{Article V – Non-Executive Roles}

\begin{enumerate}
\item[5.1] The non-executive roles will support the activities and functions of the organization, assisting the executive team in achieving the goals of AMACSS.

\item[5.2] Teams:

\begin{enumerate}
\item[5.2.1] Academics Team

\begin{enumerate}
\item[5.2.1.1] Assist in organizing and running academic-related events and seminars.

\item[5.2.1.2] Support the Vice-President Academics in communicating with students and faculty.

\item[5.2.1.3] Help manage review seminars for CMS courses.
\end{enumerate}

\item[5.2.2] Finance Team

\begin{enumerate}
\item[5.2.2.1] Assist the Vice-President Finance with financial reporting and budgeting.

\item[5.2.2.2] Help manage reimbursements and financial records.
\end{enumerate}

\item[5.2.3] Outreach Team

\begin{enumerate}
\item[5.2.3.1] Assist in creating and maintaining sponsorship packages.

\item[5.2.3.2] Help find potential sponsors and maintain relationships with external organizations.

\item[5.2.3.3] Support the organization and running of Careers and Advancement events.
\end{enumerate}

\item[5.2.4] Campus Life Team

\begin{enumerate}
\item[5.2.4.1] Assist in organizing and running campus life events.

\item[5.2.4.2] Support the Vice-President Campus Life in ensuring events cater to students' social and recreational needs.
\end{enumerate}

\item[5.2.5] Special Events Team

\begin{enumerate}
\item[5.2.5.1] Assist in organizing and running special events.

\item[5.2.5.2] Support the Vice-President Special Events in ensuring events align with AMACSS goals.
\end{enumerate}

\item[5.2.6] Marketing Team

\begin{enumerate}
\item[5.2.6.1] Assist in promoting events and seminars through social media and in-class announcements.

\item[5.2.6.2] Support the Vice-President Marketing in maintaining communications and social media accounts.
\end{enumerate}

\item[5.2.7] Technology Team

\begin{enumerate}
\item[5.2.7.1] Assist in maintaining the AMACSS website, social media pages, and mailing lists.

\item[5.2.7.2] Support the Vice-President Technology in day-to-day technology operations.

\item[5.2.7.3] Help with outreach and maintenance of open-source initiatives.
\end{enumerate}

\item[5.2.8] Operations Team

\begin{enumerate}
\item[5.2.8.1] Assist in coordinating day-to-day operational activities of the organization.

\item[5.2.8.2] Support the Vice-President Operations in managing room bookings and facility arrangements.

\item[5.2.8.3] Help coordinate with other teams to ensure smooth execution of events and activities.

\item[5.2.8.4] Assist in maintaining organizational procedures and operational guidelines.
\end{enumerate}

\item[5.2.9] First-Year Team

\begin{enumerate}
\item[5.2.9.1] Serve as liaisons between the executive and first-year students.

\item[5.2.9.2] Inform first-year students about the club's workings.

\item[5.2.9.3] Identify areas of need and communicate them to the executive.
\end{enumerate}
\end{enumerate}

\item[5.3] Team formation. It is the responsibility of the executives to design the structure of the team appropriate for the goals and objectives of the coming year.

\begin{enumerate}
\item[5.3.1] The structure of a team should including subdivisions that serves a specific function for the whole team.

\item[5.3.2] The fundamental hierarchy of a team shall include the director(s) that lead the subdivisions, as of 5.3.1, and, potentially, the associate(s) that serve directly under the respective director(s).
\end{enumerate}

\item[5.4] Hiring process. The associate members will be hired by a valid hiring committee of at least two (2) executive members, one of whom must be the President. The hiring committee must be approved by at least a majority of the executives.

\item[5.5] It is the responsibility of the executives to recruit associate members and motivate them to complete their tasks. This is a volunteer position.
\end{enumerate}

\section*{Article VI – Removal of Members and Executives}

\begin{enumerate}
\item[6.1] General Member Removal:

\begin{enumerate}
\item[6.1.1] A general member is automatically removed when they are removed from the university by the university body.

\item[6.1.2] In the case of alumni, a general member is automatically removed when their diploma is revoked.

\item[6.1.3] The organization retains the possibility to remove a general member by executive team voting at their discretion.
\end{enumerate}

\item[6.2] Exclusive Member Removal:

\begin{enumerate}
\item[6.2.1] An exclusive member is removed if they did not pay the mandatory membership fee.

\item[6.2.2] An exclusive member may be removed by executive team voting at their discretion.
\end{enumerate}

\item[6.3] Grounds for Removal:

\begin{enumerate}
\item[6.3.1] A member or executive has engaged in unlawful actions or activities;

\item[6.3.2] A member or executive has violated the Constitution;

\item[6.3.3] A member or executive has violated University of Toronto policies, procedures, or guidelines;

\item[6.3.4] A member or executive has violated the rights of a fellow member;

\item[6.3.5] A member or executive has not fulfilled their organizational responsibilities;

\item[6.3.6] Other criteria deemed to be appropriate by the Executive Committee in consultation with and approved by a majority of the general membership.
\end{enumerate}

\item[6.4] Exceptions for Removal:

\begin{enumerate}
\item[6.4.1] If a team member is unable to fulfill their responsibilities due to being on an out-of-town co-op, they cannot be removed.

\item[6.4.2] If a team member is unable to fulfill their responsibilities due to a family, medical emergency, or mental health emergency, they cannot be removed.

\item[6.4.3] There will be a time limit for the executive's absence before they are removed from the organization. This time limit is at the discretion of the President and Vice-Presidents.

\item[6.4.4] The President and Vice-Presidents will discuss who will cover the executive's role during their absence.
\end{enumerate}

\item[6.5] Executive Team Removal Procedures:

\begin{enumerate}
\item[6.5.1] President and Executive Vice-President Removal:

\begin{enumerate}
\item[6.5.1.1] The President and Executive Vice-President are removed by general referendum with a majority of 3/5 of the general membership.

\item[6.5.1.2] The referendum must be called by a petition bearing the signatures of at least 20\% of the general membership.

\item[6.5.1.3] Notice of the referendum must be given at least 14 days in advance.
\end{enumerate}

\item[6.5.2] Other Executive Team Member Removal:

\begin{enumerate}
\item[6.5.2.1] Removal is initiated by the President and Executive Vice-President.

\item[6.5.2.2] The removal must be confirmed by simple majority voting in the executive team.

\item[6.5.2.3] The executive member facing removal is entitled to vote on the motion.
\end{enumerate}
\end{enumerate}

\item[6.6] Non-Executive Team Member Removal:

\begin{enumerate}
\item[6.6.1] The decision for removal of non-executive team members is made by the executive team.

\item[6.6.2] A simple majority vote of the executive team is required for removal.
\end{enumerate}

\item[6.7] Removal Process:

\begin{enumerate}
\item[6.7.1] When removal is being considered, the member or executive must be notified in writing of the grounds for potential removal.

\item[6.7.2] The member or executive must be given an opportunity to respond to the allegations before any removal vote.

\item[6.7.3] All removal decisions must be documented and maintained for reference purposes.
\end{enumerate}
\end{enumerate}

\section*{Article VII – Finances}

\begin{enumerate}
\item[7.1] The funds of the organization shall be expended pursuant to the operating budget approved by the President and the Executive Vice-President.

\item[7.2] All Budgets shall be prepared by the Vice-President Finance in accordance with the organization's priorities as determined by the executive committee in consultation with general members at a valid general meeting.

\item[7.3] The Vice-President Finance shall present a proposed operating budget for the next fiscal year to the general membership for its consideration at the final general meeting.

\item[7.4] The operating budget shall be the major budget for the fiscal year and provide for all expenditures of the organization for the subsequent year.

\item[7.5] The banking business of the organization, or any part thereof, shall be transacted with such bank, trust company or other firm or body corporate as the Executives may designate.

\item[7.6] Officers or other persons as the Executive may designate, direct or authorize from time to time and to the extent thereby provided.

\item[7.7] The President and the Vice-President Finance shall be the sole signing authorities of banking instruments for the organization.

\begin{enumerate}
\item[7.7.1] Executives will be required to submit funding approval requests whenever funds are needed. This request must be submitted a minimum of 5 business days ahead of time
\end{enumerate}

\item[7.8] AMACSS will ensure that proper and accurate financial records are maintained and passed on to incoming executives following each year's elections.

\item[7.9] AMACSS will accept full financial and production responsibility for all activities it sponsors, plans, or executes.

\item[7.10] A financial report must be made available to the general members on a monthly or semesterly basis (decided by the President) detailing profits, expenses, and transactions that AMACSS has made within that period. These reports should be easily accessible through a medium such as the AMACSS website.
\end{enumerate}

\section*{Article VIII – General Meetings}

\begin{enumerate}
\item[8.1] The purpose of General Meetings is to provide a forum for executives to overview the activities of the organization and solicit feedback from members, to engage in policy-making, to propose amendments to the constitution, and to report on the financial status of the organization.

\item[8.2] Calling Meetings

\begin{enumerate}
\item[8.2.1] An annual general meeting is required each March, which may coincide with elections as in Article X.

\item[8.2.2] Additional general meetings may be called at the discretion of the executive.

\item[8.2.3] A notice of any general meeting must appear on the association's website and must be distributed through a general mailing list. Additional publicity is encouraged.

\item[8.2.4] Members can petition for meetings; 20 signatures require the executive to schedule a meeting within a month. 

\item[8.2.5] Any motion that is indicated on this petition of members shall be automatically included in the agenda for the general meeting.
\end{enumerate}

\item[8.3] General Meeting Agenda

\begin{enumerate}
\item[8.3.1] The executive normally determines the agenda for a general meeting.

\item[8.3.2] All executives are expected to make brief progress reports on their activities at every general meeting.

\item[8.3.3] Items for discussion at a general meeting must be circulated with the original notice of meeting

\item[8.3.4] Items may be added to the agenda, from the floor, with a simple majority. Motions to remove an executive are not valid from the floor.
\end{enumerate}

\item[8.4] Conduct of Business

\begin{enumerate}
\item[8.4.1] The President shall normally chair a general meeting. The President may request an external chair and may also be forced to do so by a majority vote of the executive prior to the meeting date.  

\item[8.4.2] General meetings will be facilitated by a Chairperson selected by the general membership from the executive committee.
The Chairperson shall be responsible for:

\begin{enumerate}
\item[8.4.2.1] Formulating and distributing an agenda for each meeting no later than two (2) days before the meeting;

\item[8.4.2.2] Ensuring appropriate conduct and leading the meeting in an efficient, reasonable manner;

\item[8.4.2.3] Moderating the discussion at meetings according to the agenda;

\item[8.4.2.4] Suspending members from participating in meetings for constitutional or procedural violations.
\end{enumerate}

\item[8.4.3] Quorum for the conduct of business at a general meeting shall be twelve (12) members.

\item[8.4.4] Each member of the organization shall be entitled to one (1) vote at a general meeting except the
Chairperson who shall only vote in the event of a tie.

\item[8.4.5] Any question at a valid general meeting shall be decided by a show of hands.

\item[8.4.6] Whenever a vote by show of hands occurs, a declaration by the Chairperson that the vote upon the question has been carried, carried by a particular majority, or failed shall be recorded in the minutes of the meeting.

\item[8.4.7] In case of an equality of votes at a valid general meeting, the Chairperson of the meeting shall have the deciding vote.

\item[8.4.8] Members are eligible to hold up to five (5) proxies from other members. 

\item[8.4.9] The chair of the meeting, or designate, will be responsible for collecting and verifying notices of proxy.

\item[8.4.10] Proxies are valid only for the conduct of business at the general meeting and shall bear no relation to any elections within the association, even if those elections are held in connection with the general meeting.

\item[8.4.11] For the conduct of elections at a general meeting neither the President nor any other member is eligible to chair, for the duration of those elections, if he or she is to be a candidate.

\item[8.4.12] The quorum requirement of twelve members, present either physically or by proxy, shall not apply to elections.

\item[8.4.13] The procedure at meetings of members shall be governed in accordance with the process outlined in
Appendix A.
\end{enumerate}

\item[8.5] Minutes

\begin{enumerate}
\item[8.5.1] A record of minutes from each general meeting shall be made publicly available to all members.

\item[8.5.2] Minutes of all general meetings must be recorded and maintained for reference purposes.
\end{enumerate}
\end{enumerate}

\section*{Article IX – Executive Meetings}

\begin{enumerate}
\item[9.1] The purpose of executive meetings is to provide a forum for the organization's executives to discuss and make decisions on day-to-day matters affecting the organization.

\item[9.2] Executive meetings will be facilitated by the President and Executive Vice-President of the organization. The President and Executive Vice-President shall be responsible for:

\begin{enumerate}
\item[9.2.1] Formulating and distributing an agenda for each meeting;

\item[9.2.2] Ensuring appropriate conduct and leading the meeting in an efficient, reasonable manner;

\item[9.2.3] Moderating the discussion at meetings according to the agenda.
\end{enumerate}

\item[9.3] The executive will meet regularly throughout the year.

\item[9.4] The frequency of executive meetings occurring between May 1 and August 31 will be left to the discretion of the executive committee.

\item[9.5] Executives are required to attend meetings and respond to requests regarding their availability in order to schedule such meetings.

\item[9.6] The executive may meet on additional occasions at the discretion of the President or any two members of the executive.

\item[9.7] Executive meetings are restricted to executive members only.

\item[9.8] Quorum of any executive meeting shall be established by the presence of a simple and clear majority of the total executives for the organization.

\item[9.9] Minutes of all executive meetings must be recorded and maintained for reference purposes.

\item[9.10] A summary of business conducted at any meeting of the executive shall be made publicly available to all members.

\item[9.11] Each member of the organization shall be entitled to one (1) vote at a valid executive meeting.

\item[9.12] Any question at an Executive Meeting shall be decided by a show of hands.

\item[9.13] Whenever a vote by show of hands occurs, a declaration by the President that the vote has been carried, carried by a particular majority, or failed shall be recorded in the minutes of the meeting.

\item[9.14] In case of an equality of votes at an Executive Meeting, the motion will be recorded as having failed.

\item[9.15] The President may, with the consent of the majority of executives, decide to adjourn these meetings from time to time.
\end{enumerate}

\section*{Article X – Emergency Meetings}

\begin{enumerate}
\item[10.1] Emergency meetings can be called for extenuating or unforeseen circumstances that may arise from time to time.

\item[10.2] These meetings must abide by the respective rules outlined in Articles VII \& VIII depending on the nature of the meeting.

\item[10.3] Notice of these meetings must be provided a minimum of 24 hours in advance through e-mail.

\item[10.4] Less notice for emergency meetings may be provided at the discretion of the President in agreement with a minimum of five (5) general members.
\end{enumerate}

\section*{Article XI – Elections and Hiring of Executives}

\begin{enumerate}
\item[11.1] Elections

\begin{enumerate}
\item[11.1.1] Each academic year, President and Executive Vice-President positions are elected positions in the association. All other executive positions are hired positions. The general members are eligible to run for election to fill these two elected positions in the following academic year. Any candidate running for the position of President must have previously served as an executive member of the association.

\item[11.1.2] A notice of elections will be made by the President no later than February 28th and shall include the date and time of the election period and whether elections will be conducted online or in-person. This notice must appear on the association's website and must be distributed through a general mailing list. Additional publicity is encouraged.

\item[11.1.3] Elections will be held in March of each academic year, no sooner than two weeks following the original notice. Elections may be conducted either online or in-person at the discretion of the executive team.

\item[11.1.4] For online elections, voting will be conducted through a secure online platform accessible to all CMS students. For in-person elections, every general member is entitled to run for either the President or Executive Vice-President position for which he/she is eligible. Elections will be held in sequence, beginning with the President followed by the Executive Vice-President. Candidates may declare their interest in any position at any time up until the election for that position takes place.
Each candidate will then have an opportunity to speak about his or her qualifications and intentions, either through a recorded video (for online elections) or at the meeting (for in-person elections).

\item[11.1.5] Each member of the association may participate in the election for each position and is entitled to cast one vote for each position.

\item[11.1.6] The candidate who receives the most votes for each position wins that position. Candidates who are not elected may run for the other elected position if it remains unfilled.

\item[11.1.7] In the case that there may be vacant elected positions they may be filled through by-election under similar conditions at any time with appropriate notice.

\item[11.1.8] If an error in the process is found, the election should be re-held with a new election oversight committee.

\item[11.1.9] Candidates who run for a position unopposed must receive two-thirds majority of the total eligible votes to be declared the winner of that election.

\item[11.1.10] Quorum for elections shall be twelve members.

\item[11.1.11] In the case that no person from the executive team wants to run for presidency, the executive team can nominate a general member to attend as a candidate. For Executive Vice-President position, any general member can run as a candidate.
\end{enumerate}

\item[11.2] Voting Process

\begin{enumerate}
\item[11.2.1] For in-person elections, all votes held in presence at the general membership shall be closed ballot forms. For online elections, voting will be conducted through a secure online platform that ensures anonymous voting while restricting access to only those who have confirmed their General Member status.

\item[11.2.2] There must be a general discussion held prior to any vote occurring, either in-person or through online forums.

\item[11.2.3] For in-person elections, all ballots must be folded once and placed in a ballot box. For online elections, votes must be submitted through the designated secure platform.

\item[11.2.4] Counting of Ballots

\begin{enumerate}
\item[11.2.4.1] For in-person elections, ballots must be counted immediately following the vote in the presence of the general membership. For online elections, results must be verified and announced within 24 hours of the election closing.

\item[11.2.4.2] A recount must be requested immediately following the initial count.

\item[11.2.4.3] In the event of a tie or a one vote difference between candidates a recount must occur.

\item[11.2.4.4] If there is still a tie following a recount, a new vote will take place.
\end{enumerate}
\end{enumerate}

\item[11.3] Hiring Executive

\begin{enumerate}
\item[11.3.1] A notice of hiring will be made by the President no later than July 1st and shall include the details of all hired positions. All executive positions except the President and Executive Vice-President are hired positions. This notice must appear on the association's website and must be distributed through social media or a general mailing list. Additional publicity is encouraged, in order to reach as much of the general student population as possible.

\item[11.3.2] A hiring committee shall be appointed by the incoming President to interview candidates for hired positions. The composition, size and guidelines of this committee are at the discretion of the incoming President, but must be approved by a simple majority of executives.

\item[11.3.3] The hiring committee shall conduct interviews as directed by the incoming President, and present their choices of executives for omnibus ratification to the first executive meeting of the new executive term, i.e., the first executive meeting after March 31st.

\item[11.3.4] The only grounds for failing to ratify the appointments of the hiring committee shall be irregularities in the hiring process. In this case all appointments shall be annulled and the hiring committee reformed as per Section 11.2.2.

\item[11.3.5] No person that serves on a hiring committee whose executive appointments fail to be ratified shall ever again be eligible to serve on a hiring committee.

\item[11.3.6] The newly ratified candidates shall be considered executives of the association immediately upon completion of the ratification process.
\end{enumerate}

\item[11.4] Leadership Changes: In the case that there is no standing President, the Executive Vice-President shall assume the role of President. If there is no Executive Vice-President, given three Vice Presidents (Academics, Campus Life, Technology) they may carry forward without a standing president in the following suggested way:

\begin{enumerate}
\item[11.4.1] VP Academics oversees academic reps, seminars, room bookings, marketing seminars and communication with academic department and professors.

\item[11.4.2] VP Campus Life oversees events team, marketing team, design team and all activities related to organizing and marketing events

\item[11.4.3] VP Technology oversees IT Team and maintains website, online communication and newsletters

\item[11.4.4] The VP Finance and VP Outreach will report to the three standing Vice Presidents.

\item[11.4.5] All Departmental Student Association matter and administrative responsibilities in terms of relations and communication with the CMS Department, Department of Student Life and SCSU is evenly split between the three Vice-Presidents.

\item[11.4.6] The three Vice-Presidents must come up with an agreement to split roles and responsibilities at the beginning of their term and clearly communicate the work split to their team.
\end{enumerate}

\item[11.5] Online Elections Process: When elections are conducted online, the following guidelines must be followed:

\begin{enumerate}
\item[11.5.1] The elections will be conducted using a secure online platform that ensures anonymous voting while restricting access to only those who have confirmed their enrollment in the CMS department.

\item[11.5.2] For every round of voting, a secure link will be provided for participants to submit their vote, which they can only access using their verified student email addresses.

\item[11.5.3] The voting process must allow for anonymous voting while ensuring voters are restricted to only those who have confirmed their CMS student status.

\item[11.5.4] All candidates must be given the opportunity to present their qualifications and intentions through recorded video presentations that will be made available to all eligible voters.

\item[11.5.5] The election oversight committee must verify the integrity of the online voting process and ensure all votes are properly counted and recorded.
\end{enumerate}
\end{enumerate}

\section*{Article XII – Amendments}

\begin{enumerate}
\item[12.1] The organization may, by resolution passed by simple majority of the general membership, make, amend or repeal this constitution or certain sections therein.

\item[12.2] Notice of a meeting called to consider such a resolution shall be given as follows:

\begin{enumerate}
\item[12.2.1] Notice of the full text of the proposed constitutional amendment shall be given to each member at least fourteen (14) days prior to the date of the meeting called to consider the change;

\item[12.2.2] A summary of the rationale for the proposed amendment shall be given to each member at least fourteen (14) days prior to the date of the meeting called to consider the change.
\end{enumerate}

\item[12.3] Amendments to the constitution require the approval of a simple majority of the members present at a valid general meeting (a general meeting that has achieved quorum).

\item[12.4] The general membership must have the final say on amendments to the Constitution.
\end{enumerate}

\section*{Article XIII – Transition}

\begin{enumerate}
\item[13.1] All outgoing executives are required to transfer all organizational resources used relative to a particular role over the course of the preceding year to new executives upon leaving the position.

\item[13.2] All outgoing executives are responsible for providing a detailed report to incoming executives that stipulates the status of ongoing projects in their portfolio and evaluations of previous projects and programs that they lead

\item[13.3] All outgoing and incoming executives will participate in a joint training session occurring no later than the end of May each year to assist with the transition between new executive teams.
\end{enumerate}

\section*{Article XIV – Handling of Food on Campus}

\begin{enumerate}
\item[14.1] The association will conform to Provincial and Municipal Health Regulations when events held at the University of Toronto at Scarborough include the sale and/or service of food items.
\end{enumerate}

\section*{Article XV – Precedence of University Policies}

\begin{enumerate}
\item[15.1] AMACSS will abide by all pertinent University of Toronto policies, procedures, and guidelines. Where the University's policies, procedures, and guidelines conflict with those of AMACSS, the University's policies, procedures, and guidelines will take precedence.
\end{enumerate}

\section*{Article XVI – Legal Liability}

\begin{enumerate}
\item[16.1] The University of Toronto at Scarborough does not endorse AMACSS's beliefs or philosophy nor does it assume legal liability for the group's activities on or off campus.
\end{enumerate}

\section*{Article XVII – Banking}

\begin{enumerate}
\item[17.1] AMACSS agrees to provide the name of the bank, the branch number and address, transit number, bank account number, and a list of all signing officers for all bank accounts opened in the organization's name to the Department of Student Life.

\item[17.2] The outgoing executive team each academic year will completely transfer all bank accounts opened in the organization's name to the new executive team by permitting the removal of the names of the current signing officers from the accounts, and forwarding all banking materials such as checkbooks to the incoming executives.
\end{enumerate}

\section*{Appendix A: General Meeting Rules of Order}

\subsection*{I – Call to Order}

\begin{enumerate}
\item[I.1] The Chairperson may call the meeting to order only if a quorum of executives and non-executive general members is present in person. If a quorum does not exist, the meeting is not qualified to conduct business. A general member may not appear by proxy
or mail ballot.

\item[I.2] The meeting must be open to all applicable general members. General members must receive notice of the meeting in accordance with the Constitution.
\end{enumerate}

\subsection*{II – Review of the Agenda}

\begin{enumerate}
\item[II.1] The first draft of the agenda is prepared by the Chairperson prior to the meeting. Agenda items should ordinarily appear in the order set forth in these rules of order.

\item[II.2] The agenda belongs to all general members. The agenda may be modified only by a majority vote. This power should only be used when necessary as proper functioning of meetings and the organization requires advance planning.

\item[II.3] At this point in the agenda, general members may add or delete items from the agenda and may change the order of presentation.

\item[II.4] When possible, changes to the agenda should be done by acquiescence of all general members. Formal voting on the agenda is only necessary where it appears to the Chairperson that there is a disagreement.
\end{enumerate}

\subsection*{III – Approval of Previous Minutes}

\begin{enumerate}
\item[III.1] The minutes need not be read aloud but they should be entered into the organization's official minute ledger upon approval by the general membership.

\item[III.2] The minutes are prepared by either the secretary or some other individual appointed by the general membership to act as recording secretary. Any general member may suggest changes to the minutes before the general membership adopts them. The suggested changes should be set forth in the minutes for the record, and then the general membership should adopt or reject such changes.

\item[III.3] Minutes should state precisely each motion considered by the general membership, and identify the general members voting in favour, against, or abstaining, and whether the motion was carried. Minutes need not reflect the comments made except in those instances when the member desires to make his/her comments recorded.

\item[III.4] When possible, changes to the minutes and adoption of the minutes should be done by acquiescence of all general members. Formal voting on the minutes is only necessary where it appears to the Chairperson that there is a disagreement.
\end{enumerate}

\subsection*{IV – Executive Reports}

\begin{enumerate}
\item[IV.1] Executives may report their findings or recommendations to the general membership at this point of the agenda.

\item[IV.2] The full report should be presented and then general members, in turn, may ask questions or comment. It is not appropriate to make motions or discuss items of business during this portion of the meeting.

\item[IV.3] This time should also be used for any presentations to be made to the general membership.
\end{enumerate}

\subsection*{V – Open Forum}

\begin{enumerate}
\item[V.1] It is the custom and practice of most organizations to allow general members an open forum to ask questions and speak about their concerns to an executive after a report has been provided.

\item[V.2] Strict time limitations should be imposed by the Chairperson and these limitations must be enforced. Each general member should address the Chairperson regarding an issue and must speak courteously and to the point.
\end{enumerate}

\subsection*{VI – Old and New Business}

\begin{enumerate}
\item[VI.1] All items that were tabled during previous meetings must be revisited during the business portion of the agenda occurring after executive reports.

\item[VI.2] The general membership may vote to postpone consideration of any old business or it may remove any item from consideration.

\item[VI.3] Except in the case of emergency business, all new items of business are heard only after all of the old items have been addressed by the general membership.

\item[VI.4] All business must be conducted in the form of motions or resolutions adopted by a vote of the general membership.
\end{enumerate}

\subsection*{VII – Motions and Deliberations}

\begin{enumerate}
\item[VII.1] When an item of business is to be discussed, the Chairperson announces the item to be discussed and opens the floor to discussion.

\item[VII.2] No general member may speak until recognized by the Chairperson. No general member may interrupt the speaker who has the floor.

\item[VII.3] The Chairperson may impose reasonable time limitations. All time limitations must be uniformly imposed upon all of the general members. The speaker shall be given a one-minute warning before time runs out. By vote of a majority of the general membership, time limits may be extended.

\item[VII.4] The Chairperson is to recognize each general member in turn. Discussion shall be limited to the item of business at hand, and the Chairperson shall have the authority to take the floor from a speaker who does not limit discussion to the item of business at hand.

\item[VII.5] No general member may speak to an issue for a second time until all other general members have had the opportunity to speak to it for the first time. Likewise, no general member may speak to an issue for a third time until all other general members have had the opportunity to speak to it for a second time.

\item[VII.6] When it appears to the Chairperson that all general members have had the opportunity to fully discuss the matter at hand, the Chair should announce that the item of business is ready for a vote.
\end{enumerate}

\subsection*{VIII – Voting}

\begin{enumerate}
\item[VIII.1] There are 3 basic motions for each item of business:

\begin{enumerate}
\item[VIII.1.1] A motion to adopt a specific action by the board.

\item[VIII.1.2] A motion to postpone the item to another meeting (including fact-finding assignments to a person or committee).

\item[VIII.1.3] A motion to remove an item from consideration.
\end{enumerate}

\item[VIII.2] The general membership is limited to discussing one item of business at a time, but there are no limits to the number of motions that may be considered as to how to dispose of that item of business.

\item[VIII.3] After the general membership has had the opportunity to discuss each motion presented for consideration, the Chairperson will call each motion presented to a vote.

\item[VIII.4] The fact that a motion has been adopted or failed does not prevent the item of business from being added to the agenda in the future and all motions may be reconsidered at any time by the general membership.
\end{enumerate}

\end{document} 